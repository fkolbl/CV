%-------------------------------------------------------------------------------
%	SECTION TITLE
%-------------------------------------------------------------------------------

\cvsection{Thèses en cours et soutenues}
%-------------------------------------------------------------------------------
%	CONTENT
%-------------------------------------------------------------------------------
\begin{cvskills}
%---------------------------------------------------------
  \cvskill
    {L. Regnacq} % Category
    {\begin{itemize}
    	\item Financement : Projet joint ANR-NIH BIOTIFS,
    	\item Co-encadrement : O. Romain (ETIS), Y. Bornat (IMS),
    	\item sujet de thèse : Improving the selectivity of peripheral nervous system electrical stimulation using Intrafascicular electrodes and non-conventional waveforms,
    	\item date de soutenance : 6 sepmtebre 2023.
    \end{itemize}
    } % Skills           
  \cvskill
    {T. Couppey} % Category
    {\begin{itemize}
    	\item Financement : Bourse Ecole Doctorale,
    	\item Co-encadrement : O. Romain (ETIS), O. Français (ESYCOM),
    	\item sujet de thèse : Modélisation et conception d'un banc de mesure de tomographie électrique d'impédance pour la localisation d'activité dans le système nerveux périphérique,
    	\item date de soutenance prévue : septembre 2024.
    \end{itemize}
    } % Skills    
  \cvskill
    {L. Lecomte} % Category
    {\begin{itemize}
    	\item Financement: bourse CIFRE (ANRt), collaboration avec l'entreprise \href{http://fineheart.fr}{FineHeart},
    	\item Co-supervision: N. Lewis (IMS), M. Maldari (FineHeart), S. Garrigue (FineHeart)
    	\item Focus: extraction de donnée physiologiques par medure de cardio-impedance,
    	\item date de soutenance prévue : janvier 2027.
    \end{itemize}
    } % Skills    
\end{cvskills}

\cvsection{Participation à des jurys de thèse}
\begin{cvskills}
\cvskill
    {Houssein Mariam} % Category
    {Caractérisation hyperfréquence par spectroscopie diélectrique de composés biologiques en environnement microfluidique. Thèse de l'Université Paris Est, soutenue le 16/12/2020 sous la direction d'O. Français et d'E. Richalot.
    } % Skill
\cvskill
    {Farad Khoyratee} % Category
    {Conception d’une plateforme modulable de réseaux de neurones biomimétiques pour l’étude des maladies neurodégénératives. Thèse de l'Université de Bordeaux, soutenue le 13/12/2019 sous la direction de S. Saïghi et de T. Lévi.
    } % Skill
\end{cvskills}